
\documentclass[12pt]{article}
\usepackage{fancybox}
\usepackage{fullpage}
\usepackage{psfig}
\usepackage{url}
%\newcommand{\onef}[3]{\psfig{figure=#1,width=#2,height=#3}}
\usepackage{xcolor}
\usepackage{bm}
\usepackage{color, colortbl}
%\definecolor{lightgray}{gray}{0.9}
%---------------------------------
\newcommand{\be}{\begin{eqnarray}}
\newcommand{\ee}{\end{eqnarray}}
  \newcommand{\ba}{\begin{eqnarray*}}
\newcommand{\ea}{\end{eqnarray*}}

\newcommand{\onef}[3]{
\begin{center}~
{\psfig{figure=#1,width=#2,height=#3}}
\vskip -0.15truein
\end{center}}

%___
%--------------------------------------------
\newcommand{\twof}[4]
{
\hbox to\hsize{\hss
    \vbox{\psfig{figure=#1,width=#3,height=#4}}\qquad
    \vbox{\psfig{figure=#2,width=#3,height=#4}}
    \hss}
}


%---------------------------------
\usepackage[framemethod=TikZ]{mdframed}
\usepackage{lipsum}
\mdfdefinestyle{MyFrame}{%
    linecolor=black,
    outerlinewidth=1.5pt,
    roundcorner=5pt,
    innertopmargin=\baselineskip,
    innerbottommargin=\baselineskip,
    innerrightmargin=20pt,
    innerleftmargin=20pt,
    backgroundcolor=gray!20!white}


\begin{document}
\thispagestyle{empty}
%\addtocounter{page}{-1}
\vspace*{0.1in}

\begin{center}
        { \Large \bf   5.4 EXERCISES}
\end{center}
\noindent {\bf BMED6420}

\noindent  Brani  Vidakovic;  Fall 2018


\noindent Consult the class slides, hints, and  cited literature for
the solution of exercise problems.



\vspace{0.5in}

 \noindent {\bf 1.   Laplace's Method for Binomial-Beta.~}
Approximate the posterior by a normal model using Laplece's method.
The likelihood is binomial $X|p \sim {\cal B}in(n, p)$ with and the prior
is $p \sim {\cal B}e(\alpha, \beta) $ with parameters:

(a) $\alpha=\beta =1$ (flat prior)

(b) $\alpha = \beta = 1/2 $ (Jeffreys' prior)

(c) $\alpha = \beta = 2.$

How well the Laplace's method approximates 95\% CS for $p$?
Compare exact equi-tailed CS's with the approximations. Use $n=20$
and $X = 8.$


\vspace{0.3in}

 \noindent {\bf 2.   Coin and Probability of Heads.~}
 A coin is flipped unknown number of times, say $n$,
and $X$  heads are observed. Suppose we are interested in
probability of heads $p$ and assume
model
\ba
P(X=k|n, p) \sim {\cal B}in(n, p).
\ea
The model is completed with priors on $n$ and $p$.
We assume that $n$ is Poisson with mean $\lambda$
and $p$ is beta ${\cal B}e(\alpha, \beta).$ The priors on $n$ and
$p$ are independent.

Assume $X=6$ is observed. Sample from the posterior
using

(a) Metropolis algorithm

(b) Gibbs sampler

Assume that $\alpha = \beta= 20$ and $\lambda = 18.$

\vspace*{0.2in}

 \noindent  {\bf 3.  Amanita muscaria.~}
With its bright red, sometimes dinner-plate-sized caps,
the fly agaric ({\it Amanita muscaria})
is one of the most striking of all mushrooms (Fig.~\ref{fig:miscaspo}a).
The white warts that adorn the cap, the white gills, a well-developed ring,
and the distinctive volva of concentric rings distinguish the
fly agaric from all other red mushrooms. The spores
 of the mushroom print white,
 are elliptical, and have  a larger axis in the range of 7 to 13 $\mu$m (Fig.~\ref{fig:miscaspo}b).
%
%\begin{figure} [hbtp]
%\twof{amanitamuscaria.eps}{muscariaspores.eps}{1.5in}{1.7in}
%\caption{ (a) Fly Agaric or {\it Amanita Muscaria}; (b) Spores of {\it Amanita Muscaria}.}
%\end{figure}


\begin{figure} [hbtp]
\centering
\twof{amanitamuscaria.eps}{muscariaspores.eps}{1.0in}{1.25in}
\caption{{\it Amanita muscaria} and its spores. (a) Fly agaric or {\it Amanita muscaria}. (b) Spores of {\it Amanita muscaria}.  \label{fig:miscaspo}   }
\end{figure}




Measurements of the diameter $X$ of spores for $n=51$  mushrooms
are given in the following table:

\begin{center}
\begin{tabular}{cccccccccc}
\hline
 10 & 11 & 12 &  9 & 10 & 11 & 13 & 12 & 10 & 11 \\
 11 & 13 &  9 & 10 &  9 & 10 &  8 & 12 & 10 & 11 \\
  9 & 10 &  7 & 11 &  8 &  9 & 11 & 11 & 10 & 12 \\
 10 &  8 &  7 & 11 & 12 & 10 &  9 & 10 & 11 & 10 \\
 8 & 10 & 10 &  8 &  9 & 10 & 13 &  9 & 12 &  9 \\
 9  &    &    &    &    &    &    &    &    &    \\
  \hline
\end{tabular}
\end{center}

Assume that the measurements are normally distributed with mean $\mu$ and
variance $\sigma^2,$ but both parameters are unknown and of interest.

Suppose that the prior on $\mu$ is normal ${\cal N}(12, 2^2)$ and
the prior on $\tau  = 1/\sigma^2$ is gamma ${\cal G}a(2,4).$

(a) Develop Gibbs sampling algorithm and find Bayes estimators of $\mu$ and $\tau.$

(b) If an inverse gamma ${\cal IG}a(4,2)$ is placed on $\sigma^2$, would the solution be different?
Develop a Gibbs sampling for this case keeping the prior on $\mu$ as in (a).


(c) Develop a Metropolis algorithm for the priors as in (a). Choice of proposal distributions is up to you.


\vspace*{0.3in}

\noindent
{\bf  4. Jeremy via Metropolis.~}
The Jeremy example was solved exactly (as a conjugate problem) and
by using WinBUGS.
Recall that Jeremy's IQ test score $X$ was modeled as normal ${\cal N}(\theta, 80),$
while the location parameter $\theta $ had prior ${\cal N}(110, 120).$ We could think of $\theta$ as
Jeremy's intrinsic IQ level.
On his IQ test Jeremy scored $X=98.$

Develop Metropolis MCMC scheme for sampling from the posterior.

(i) Sample the proposal  $\theta_p$  from Cauchy distribution centered at the current
state of the chain $\theta$.

\begin{verbatim}
 thetaprop = theta  + 10*tan(pi*rand(1,1)-pi/2);
 \end{verbatim}

 Since the proposal distribution depends on $(\theta_p - \theta)^2$ it does not
 factor into the acceptance ratio $\rho.$

 Your target is the product of likelihood and prior and does not need
 to be normalized to be a distribution. Start with $\theta = 100$ and burn in 500 simulations.
 Save 50,000 simulations (not including the burn in).

 (a) Estimate posterior mean, posterior standard deviation, fine the median and
 95\% equaly-tailed credible set.

  (b) Plot the normalized histogram and superimpose the
  theoretical posterior density.


\vspace*{0.3in}

\noindent
{\bf 5. Jeremy via Gibbs Sampling.~}

Translate the following WinBUGS code into custom made Gibbs Sampling program in MATLAB.

\begin{verbatim}
 model{
 x ~ dnorm(mu, tau)
pprec <- 1/120
mu ~ dnorm(110, pprec)
tau ~ dgamma(0.01, 1)
sig2  <- 1/tau
}

DATA
list(  x = 98  )

INITS
list(mu=100, tau=0.01)
\end{verbatim}

Find full conditionals for $\mu$ and $\tau$.

Compare Bayes summaries (posterior mean, median, 2.5 and 97.5 percentiles, histograms)
between WinBUGS and Gibbs outputs.



\vspace*{0.3in}
\noindent { \bf Gibbs with Metropolis Step.~}
 %
When in Gibbs sampling scheme, one or more full conditionals are
not tractable, their function can be replaced by a so-called Metropolis step.
An example with Metropolis step is given below.

\vspace*{0.25in}
\noindent {\bf Georgia Deaths from Kidney Cancer 1985-1989 by Counties.~}
%
 Data set contains death counts from Kidney Cancer for 159 Georgia counties as well
 as the county population size. The data are from years 1985-1989.

 It is of interest to estimate the death rate (per 100,000) for each county,
 as well as the all-Georgia death rate.

 The model is Poisson ${\cal P}oi(\lambda_i n_i), i = 1,\dots, k$ where
 $n_i$ is the population size divided by 100,000 for  county $i$; here $k=169.$
 Also, $n = n_1 + \dots + n_k$ is population of Georgia in units of 100,000.

 The model is as follows:

 \ba
&&  y_i \sim {\cal P}oi(\lambda_i n_i), i = 1,\dots, k \\
&& \lambda_i \sim {\cal G}a(\alpha, \beta), i = 1,\dots, k \\
&& \alpha \sim {\cal U}(0, A)\\
&& \beta \sim {\cal U}(0, B)
\ea
for specified constants $A$ and $B$.


The joint distribution is:
\ba
f({\bm y},{\bm \lambda}, \alpha, \beta) =  \prod_{i=1}^k \left[\frac{\lambda_i ^{y_i}}{y_i!} e^{-\lambda_i n_i}
\times \frac{ \beta^\alpha {\lambda_i}^{\alpha-1}}{\Gamma(\alpha)} e^{-\beta \lambda_i}\right]
\times \frac{1}{A} {\bm 1}(0 \leq \alpha \leq A) \times  \frac{1}{B} {\bm 1}(0 \leq \alpha \leq B).
\ea

\ba
\pi(\lambda_i |{\bm y},{\bm \lambda}_{\ne i}, \alpha, \beta) &\propto&  f({\bm y},{\bm \lambda}, \alpha, \beta) \\
&\propto& \lambda_i^{y_i} e^{- \lambda_i n_i} \lambda_i^{\alpha -1} e^{-\beta \lambda_i}, i=1, \dots, k.
\ea

 Therefore,
 \ba
[ \lambda_i |{\bm y},{\bm \lambda}_{\ne i}, \alpha, \beta] ~\sim~ {\cal G}a(y_i + \alpha, n_i + \beta), ~ i=1,\dots, k.
\ea


\ba
\pi(\alpha |{\bm y},{\bm \lambda},   \beta)  &\propto&  f({\bm y},{\bm \lambda}, \alpha, \beta) \\
&\propto&  \frac{\beta^{k \alpha}}{(\Gamma(\alpha))^k}
\left( \prod_{i=1}^k \lambda_i \right)^{\alpha -1} \frac{1}{A} {\bm 1}(0 \leq \alpha \leq A) \\
&\propto& \frac{\left(\beta^k \prod_{i=1}^k \lambda_i \right)^\alpha}{(\Gamma(\alpha))^k}
  {\bm 1}(0 \leq \alpha \leq A).
 \ea


\ba
\pi(\beta |{\bm y},{\bm \lambda}, \alpha) &\propto&  f({\bm y},{\bm \lambda}, \alpha,\beta) \\
&\propto& \beta^{k \alpha} e^{- \beta \sum_{i=1}^k y_i}   \times  \frac{1}{B} {\bm 1}(0 \leq \beta \leq B).
\ea

 Therefore,
 \ba
[ \beta |{\bm y},{\bm \lambda}, \alpha] ~\sim~ {\cal G}a\left(k \alpha+1, \sum_{i=1}^k y_i \right)
 {\bm 1}(0 \leq \beta \leq B).
\ea

The Metropolis Step for generating $\alpha$:

Proposal $\alpha' \sim {\cal N}(\alpha, \sigma^2).$
The Metropolis ratio depends on target only,
\ba
r =   \left( \frac{\Gamma(\alpha)}{\Gamma(\alpha')}\right)^k \times C^{\alpha' - \alpha}
\times  {\bm 1}(0 \leq \alpha' \leq A),
\ea
for $C=\beta^k \prod_{i=1}^k \lambda_i.$


From {\tt gibbsmetstep.m}:

 \begin{figure}[h]
 \centering
 \twof{gacancer1.eps}{gacancer2.eps}{2.9in}{2.9in}
 \end{figure}


\vspace*{0.25in}
\noindent {\bf Exponential Survival Times.~}
Survival times of $n$  post-surgery patients assigned to a treatment group
are given as $ T_1, \dots,  T_n.$
In a control group of $m$   post-surgery patients
the survival times  are $C_1, \dots, C_m.$

The proposed model is
\ba
&&T_i \sim {\cal E}(\lambda \theta),~ i = 1,\dots, n\\
&&C_i  \sim {\cal E}(\lambda),~ i = 1,\dots, m\\
&&~~~~~~~~\lambda >0, \theta > 0.
\ea
where
$T_i$'s and $C_i$'s are assumed independent.

To complete the model assume the following priors on
$\lambda$ and $\theta$,
\ba
&&\pi(\lambda) =  \frac{1}{\lambda},~~~\mbox{[Jeffreys' choice]}\\
&&\theta \sim {\cal G}a(1/2, 1/2),
\ea
where gamma is parameterized via the rate parameter.

(a) Show that the likelihood and posterior depend  on $T = \sum_{i=1}^n T_i$ and
$C = \sum_{i=1}^m C_i.$

(b) Identify full conditionals for $[\lambda|\theta, T, C]$ and
$[\theta|\lambda, T, C].$

(c) For $n=10, m=12, T=18.26,$ and $C=26.78$ set Gibbs' sampling scheme.

(d) From a Gibbs sample of size 10000 estimate $\theta$ and find
95\% equitailed credible set. Is the credible set containing 1?
Discuss.

(e) How would you deal with the analysis if some times are censored? Repeat
the analysis if $T_{10}=2.12, C_{11}=1.54$ and $C_{12} = 1.98$
are in fact censored times.

\vspace*{0.3in}
\noindent{\bf Testing the Effectiveness of a Seasonal Flu Shot.~}
Assume 30 individuals are given a flu shot at the start of winter.
At the end of winter, follow up to see whether they contracted
flu.
Let
\ba
x_i = \left\{ \begin{array}{ll}
1, & \mbox{~~no flu (shot effective)}\\
0, & \mbox{~~flu (shot not effective)}
\end{array} \right.
 \ea
 Suppose the 30th individual was unavailable for followup.
Define $y = \sum_{i=1}^{29} x_i.$
 If $p$ is the probability the shot is effective, then
\ba
f(y|p) = {{29}\choose{y}}  p^y (1-p)^{29-y},~ y=0,1,\dots, 29.
\ea
With the complete data ( $y$ plus $x_{30}$),
\ba
f(y, x_{30}|p) = {{30}\choose{y+x_{30}}}  p^{y+x_{30}} (1-p)^{30-y-x_{30}},~ y+x_{30}=0,1,\dots, 30.
\ea
With uniform prior on $p$  and $[x_{30}|p] \sim {\cal B}er(p)$
the joint distribution of $[p, x_{30}|y]$ is
proportional to $f(y, x_{30}|p).$
 Thus, the full conditionals are
\ba
[p | x_{30}, y] \sim  {\cal B}e(y+x_{30} + 1, 30 - y - x_{30}+1),
\ea
and $[x_{30}|y, p] \sim {\cal B}er(p).$

If $y=21,$ estimate $p$ using Gibbs sampler that uses these two full conditionals.

\newpage
\begin{center}
        { \Large \bf   Hints/Results/Solutions to Some of the Exercises}
\end{center}

\vspace{0.3in}


 \noindent {\bf 1.   Laplace's Method for Binomial-Beta.~}

\vspace{0.3in}

 \noindent {\bf 2.   Coin and Probability of Heads.~}

{\sl Hint for (a).}  Show first that
$[X|p] \sim {\cal P}oi(\lambda p). $ To do this you will
need to integrate $n$ from the $X \sim {\cal B}in(n,p).$

\ba
P(X=k|p) = \sum_{n=k}^\infty P(X=k|n,p) \frac{\lambda^n}{n!} e^{-\lambda}.
\ea
Since $P(X=k|n,p)={{n}\choose{k}} p^k (1-p)^{n-k}$ after
change of variables $u=n-k$ and rearrangement of the sum
you will arrive to Poisson ${\cal P}oi(p \lambda)$ distribution.
 Here $k$ is constant, and you will need to use
 $\sum_{u=0}^\infty  [(1-p) \lambda]^u/u! = \exp\{(1-p) \lambda\}.$

 Now,  the posterior is proportional to the product of beta
 prior and marginal likelihood for $[X|p]$ which is the Metropolis
 target function.

 Take as proposal   beta ${\cal  B}e(X+\alpha, \beta)$ distribution.
 Note that this is an independence proposal which conveniently
 simplifies acceptance probability in Metropolis algorithm.

 Pick any $p$ between 0 and 1 and start the simulation.

 \vspace*{0.1in}

 {\sl Hint for (b).}  For Gibbs, you will need full conditionals.
 Argue that:

 \ba
 [ n | X, p ]   \sim    X + {\cal P}oi( (1-p)\lambda ).
\ea
and
\ba
 [ p | n, X ]   \sim  {\cal B}e(\alpha + X, \beta + n - X).
\ea

 The first one is simply number of heads (observed) plus number of
 tails (unobserved) and by argument similar to one in (a)
 distributed as ${\cal P}oi( (1-p)\lambda ) .$ The second follows
 from a conjugate Binomial-Beta setup.


\vspace*{0.3in}

\noindent  {\bf 3.  Amanita muscaria.~}

{\sl Hint:} Bayes' inference depends on the sufficient statistics $\overline{X} \sim {\cal N}(\mu, \frac{\sigma^2}{n}).$

\vspace{0.3in} \noindent
{\bf  4. Jeremy via Metropolis.~}

\vspace{0.3in} \noindent
{\bf 5. Jeremy via Gibbs Sampling.~}

\vspace{0.3in} \noindent
{\bf 6. Gibbs with Metropolis Step.~}

\begin{verbatim}
%
% Kidney Cancer in Georgia 1985-1989 by county
%
close all
clear all

%rand('seed',2);
%randn('seed',2);
filename = 'C:\Brani\Courses\isyestatg\ISyE6420Spring2018\HWs\georgiakcd.xlsx';


data = xlsread(filename,  'B:D');

 y = data(:,2);
 ni = data(:,3)/100000;
 [n ~]=size(data);
 A=10;
 B=10;
 %
lambda = ones(n,1); % initial values
alpha = 1;
beta=1;
sigma=0.1;
%
lambdas =[lambda]; %save all lambdas
alphas =[alpha];
betas= [beta];
%
%
tic
for i = 1:20000  %beware slow...
   prodlambdas = prod(lambda);
   sumlambdas = sum(lambda);
   % lambda_k ~ Gamma(y_k + alpha, ni_k + beta), k=1,...,n
   lambda=zeros(n,1);
   for k = 1:n
       lambda(k) = gamrnd(y(k)+alpha, 1/(ni(k)+beta));
   end
   lambdas=[lambdas  lambda];

   % METROPOLIS STEP for alpha
   alpha_prop =  alpha + sigma*randn;
   c= prodlambdas * beta^n;
   if   (alpha_prop < A) .* (alpha_prop > 0)
          r = (gamma(alpha).^n * c^alpha_prop)/(gamma(alpha_prop).^n * c^alpha) ;
   else
          r = 0;
   end
   if rand < r
       alpha = alpha_prop;
   end
   alphas=[alphas alpha];
   % RESTRICTED GAMMA for beta

   beta_new = B+1;
   while beta_new > B
        beta_new = gamrnd(n*alpha + 1, 1/sumlambdas);
   end
   beta = beta_new;
   betas = [betas beta];
  end
toc
%Burn in 500
burn=500;
 lambdas = lambdas(:, burn+1:end);
 alphas = alphas(burn+1:  end);
  betas = betas(burn+1: end);
figure(1)
nbins=30;
subplot(3,1,1)
histogram(lambdas(33,:),nbins,'Normalization','Probability')
subplot(3,1,2)
histogram(alphas,nbins,'Normalization','Probability')
subplot(3,1,3)
histogram(betas,nbins,'Normalization','Probability')
print -depsc 'gacancer1.eps')

figure(2)
subplot(3,1,1)
plot(lambdas(33,2000:2200),'-')
axis tight
subplot(3,1,2)
plot(alphas(2000:2200),'-')
axis tight
subplot(3,1,3)
plot(betas(2000:2200),'-')
axis tight
print -depsc 'gacancer2.eps')

ma=mean(alphas)
mb=mean(betas)
mla = mean(lambdas')
ma/mb
sum(y)/sum(ni)
\end{verbatim}

\vspace*{0.25in}
\noindent {\bf Exponential Survival Times.~}

{\sl Hint:}
The posterior is proportional to
\ba
\prod_{i=1}^n (\lambda \theta \exp\{-\lambda \theta T_i\}) \times
\prod_{i=1}^m (\lambda  \exp\{-\lambda  C_i\}) \times
\frac{1}{\lambda} \times \theta^{1/2 - 1} \exp\{-\theta/2\}
\ea
The above product is proportional to
\ba
\lambda^{m+n-1}\exp\{-\lambda(\theta T + C)\},
\ea
when $\theta$ is constant, and to
\ba
\theta^{n-1/2} \exp\{-(\lambda T + 1/2) \theta)\},
\ea
when $\lambda$ is constant.
Thus,
$[\lambda|\theta, T, C] \sim {\cal G}a(m+n, \theta T + C)$ and
$[\theta|\lambda, T, C] \sim {\cal G}a(n+1/2, \lambda T + 1/2).$





\end{document}
