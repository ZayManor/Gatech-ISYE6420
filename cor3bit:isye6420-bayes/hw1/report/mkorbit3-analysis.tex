\documentclass[a4 paper]{article}
% Set target color model to RGB



\usepackage[inner=2.0cm,outer=2.0cm,top=2.5cm,bottom=2.5cm]{geometry}
\usepackage{setspace}
\usepackage[rgb]{xcolor}
\usepackage{verbatim}
\usepackage{subcaption}
\usepackage{amsgen,amsmath,amstext,amsbsy,amsopn,tikz,amssymb,tkz-linknodes}
\usepackage{fancyhdr}
\usepackage[colorlinks=true, urlcolor=blue,  linkcolor=blue, citecolor=blue]{hyperref}
\usepackage[colorinlistoftodos]{todonotes}
\usepackage{rotating}
%\usetikzlibrary{through,backgrounds}

\usepackage{lmodern}
\usepackage[T1]{fontenc}
\usepackage[capposition=top]{floatrow}
\usepackage{hyperref}
\usepackage{graphicx}
\graphicspath{ {images/} }
\usepackage{booktabs}
\usepackage{changepage}
\usepackage{float}








\hypersetup{%
pdfauthor={Nick Korbit},%
pdftitle={Homework},%
%pdfkeywords={Tikz,latex,bootstrap,uncertaintes},%
pdfcreator={PDFLaTeX},%
pdfproducer={PDFLaTeX},%
}
%\usetikzlibrary{shadows}
% \usepackage[francais]{babel}
\usepackage{booktabs}


\input{macros.tex}



%%%%%%%%%%%%%%%%%%%%%%%%%%%%%%%%%%%%%%%%
%%			 Document				  %%
%%%%%%%%%%%%%%%%%%%%%%%%%%%%%%%%%%%%%%%%


\begin{document}
	
\homework{Homework \#1}{Spring 2020}{Roshan Vengazhiyil, Brani Vidakovic}{}{Nick Korbit}{903263968}


%%%%%%%%%%%%%%%%%%%%%%%%%%%%%%%%%%%%%%%%
%%			 Part I					  %%
%%%%%%%%%%%%%%%%%%%%%%%%%%%%%%%%%%%%%%%%

\problem{1}

\textbf{a$)$.} We first calculate the sensitivity and 
specificity for the serial system. As per \cite{stat}:
$$
\begin{aligned}
&\mathrm{Se}=\mathrm{Se}_{1} \times \mathrm{Se}_{2} \times \cdots \times \mathrm{Se}_{k}\\
&\mathrm{Sp}=1-\left[\left(1-\mathrm{Sp}_{1}\right) \times\left(1-\mathrm{Sp}_{2}\right) \times \cdots \times\left(1-\mathrm{Sp}_{k}\right)\right]
\end{aligned}
$$

We now combine three tests -- Tinel's sign (TS), 
Phalen's test (PH) and
the nerve conduction velocity test (NCV). So that:

$$
Se = 0.97 \times 0.92 \times 0.93 = ‭0.829932‬
$$

$$
Sp = 1 - (1 - 0.91) \times (1 - 0.88) \times (1 - 0.87) = 0.998596
$$


\textbf{b$)$.} We then turn to the parallel systems. 
As per \cite{stat}:

$$
\begin{aligned}
&\mathrm{Se}=1-\left[\left(1-\mathrm{Se}_{1}\right) \times\left(1-\mathrm{Se}_{2}\right) \times \cdots \times\left(1-\mathrm{Se}_{k}\right)\right]\\
&\mathrm{Sp}=\mathrm{Sp}_{1} \times \mathrm{Sp}_{2} \times \cdots \times \mathrm{Sp}_{k}
\end{aligned}
$$

We combine three tests:

$$
Se = 1 - (1 - 0.97) \times (1 - 0.92) \times (1 - 0.93) = 0.999832 ‭
$$

$$
Sp = 0.91 \times 0.88 \times 0.87 = 0.696696
$$




\textbf{c$)$.} After finding $Sp$ and $Se$ we calculate $PPV$
for both serial and parallel systems. As per \cite{stat}:
$$
\mathrm{PPV}=\frac{\text { Se } \times \text { Pre }}{\text { Se } \times \text { Pre }+(1-\mathrm{Sp}) \times(1-\text { Pre })}
$$

We first find $Pre$ as $50/1000=0.05$. The we find

$$
\mathrm{PPV_{serial}}=\frac{‭0.829932‬ \times 0.05}
{‭0.829932 \times 0.05 + (1-0.998596) \times (1-0.05)} \approx 0.96885857
$$

And 

$$
\mathrm{PPV_{parallel}}=\frac{0.999832‬ \times 0.05}
{0.999832 \times 0.05 + (1-0.696696) \times (1-0.05)} \approx 0.14784710
$$





\problem{2}

For the second problem we are basically to build 
a Naive Bayes classifier. We start with calculating 
priors -- $P(Class='Went Beach')=40/100=0.4$ 
and $P(Class=not'Went Beach')=1-0.4=0.6$. 

Then we go to the ``train'' phase. We calculate 
conditional probabilities 
for each feature -- 'Midterm', 'Finances', 
'Friends Go', 'Forecast' and 'Gender'. That's 
easy as all features have binary output, so
we just calculate means for cases 'Went Beach'=True 
and 'Went Beach'=False. We cache those values.

In the the ``predict'' phase we multiply all 
conditional probabilities depending on each 
person parameters and then multiply again by 
the prior. We do that for both  
'Went Beach'=True and 'Went Beach'=False
classes. Then we normalize the output.

Let's test our classifier for three hypothetical 
persons -- Jane, Michael and Melissa. We can 
parameterize each one with a dictionary:

\begin{verbatim}
    jane = {
'Midterm': 1,
'Finances': 1,
'Friends Go': 0,
'Forecast': 0,
'Gender': 1,
}
\end{verbatim}	 

Then we run ``predict'' phase and 
analyze the output:
\begin{itemize}
	\item Jane, True: 0.17238060388100107, False: 0.8276193961189989
	\item Michael, True: 0.4070417547103887, False: 0.5929582452896114
	\item Melissa, True: 0.2796420404874101, False: 0.7203579595125899
\end{itemize}

We see that Jane's output matches the HW1 example and
the highest probability to go to the beach belongs to Michael 
- around 40\%.

Note. Both code (hw1q2.py) and data (naive.csv) are included 
in the zip archive. The only code dependency is pandas package.
To run the code just run \textit{'python hw1q2.py'}. 
Tested for Python 3.8.


\problem{3}

Let's first parameterize the problem. We 
introduce two binary variables -- 'Knowledge' ($K$) 
and 'Question' ($Q$). Knowledge is set to True if a 
student knows the answer to the question, otherwise 
it's False. Question is True if an answer to the given 
question is correct, otherwise it's False. We can 
represent the system via causality $K \rightarrow Q$,
so that $Q$ depends from $K$. We are also given the 
probabilities: $P(K)=0.8$, $P(Q|K)=1.0$ and 
$P(Q|\neg K)=0.25$. \newline 


\textbf{a$)$+c$)$.} As the questions are independent 
let's first find the probability of one question 
to be answered correctly or, more formally,
let's find the total probability $P(Q)$:

$$
P(Q) = P(Q|K) \times P(K) + P(Q|\neg K) \times P(\neg K)  
$$

So that
$$
P(Q) = 1.0 \times 0.8 + 0.25 \times 0.2 = 0.85 
$$

If we have $n$ independent questions then 
the probability of \textit{all} the questions
to be answered correctly will be calculated in 
the serial manner: $P_n=P(Q)^n$. If $P(Q)<1$ then 
with $n \rightarrow \infty$ we have $P_n \rightarrow 0$.

In the case of $n=2$, we have $P_2=0.85^2=0.7225$. \newline

\textbf{b$)$+c$)$.} Let's then find a probability of 
Knowledge if a question was answered correctly -- $P(K|Q)$:

$$
P(K|Q) = \frac{P(Q|K) \times P(K)}{P(Q)}
$$

We have already calculated $P(Q)$, so let's find $P(K|Q)$: 

$$
P(K|Q) = \frac{1.0 \times 0.8}{0.85} \approx 0.94117647
$$

If we have $n$ independent questions then 
the probability of having known \textit{all} 
the questions will be calculated in 
the serial manner: $P_n=P(K|Q)^n$. If $P(K|Q)<1$ then 
with $n \rightarrow \infty$ we have $P_n \rightarrow 0$.

In the case of $n=2$, we have $P_2=0.94117647^2 \approx 0.88581315$.


%%%%%%%%%%%%%%%%%%%%%%%%%%%%%%%%%%%%%%%%
%%			 Bibliography			  %%
%%%%%%%%%%%%%%%%%%%%%%%%%%%%%%%%%%%%%%%%

\begin{thebibliography}{9}


\bibitem{stat}\label{stat} 
Engineering Biostatistics: An Introduction using MATLAB and WinBUGS. 
Brani Vidakovic - Wiley Series in Probability and Statistics.

\end{thebibliography}



\end{document} 