
\documentclass[12pt]{article}
\usepackage{fancybox}
\usepackage{fullpage}
\usepackage{psfig}
\usepackage{url}
%\newcommand{\onef}[3]{\psfig{figure=#1,width=#2,height=#3}}
\usepackage{xcolor}
\usepackage{color, colortbl}
%\definecolor{lightgray}{gray}{0.9}

\newcommand{\onef}[3]{
\begin{center}~
{\psfig{figure=#1,width=#2,height=#3}}
\vskip -0.15truein
\end{center}}
%___
%--------------------------------------------
\newcommand{\twof}[4]
{
\hbox to\hsize{\hss
    \vbox{\psfig{figure=#1,width=#3,height=#4}}\qquad
    \vbox{\psfig{figure=#2,width=#3,height=#4}}
    \hss}
}


%---------------------------------
\usepackage[framemethod=TikZ]{mdframed}
\usepackage{lipsum}
\mdfdefinestyle{MyFrame}{%
    linecolor=black,
    outerlinewidth=1.5pt,
    roundcorner=5pt,
    innertopmargin=\baselineskip,
    innerbottommargin=\baselineskip,
    innerrightmargin=20pt,
    innerleftmargin=20pt,
    backgroundcolor=gray!20!white}


 \newcommand{\be}{\begin{eqnarray}}
\newcommand{\ee}{\end{eqnarray}}
\newcommand{\ba}{\begin{eqnarray*}}
\newcommand{\ea}{\end{eqnarray*}}




\begin{document}
\thispagestyle{empty}
%\addtocounter{page}{-1}
\vspace*{-0.1in}

\begin{center}
        { \Large \bf   7.4 EXERCISES}
\end{center}
\noindent {\bf BMED6420}

\noindent  Brani  Vidakovic;  Fall 2018


\noindent Consult the class slides, hints, and  cited literature for
the solution of exercise problems.

%various models
\vspace*{0.2in}
\noindent  {\bf 1.  Beetles (Bliss Data).~}
In his 1935 paper\footnote{
Bliss, C. I. (1935).
The calculation of the dose-mortality curve. {\it Ann. Appl. Biol.}, {\bf 22}, 134--167.},
 Bliss provides a table
showing a number of flour beetles killed after 5 hours of exposure to
gaseous carbon disulfide at various concentrations.
 This data set has since been used extensively
by statisticians to illustrate and compare models for binary and binomial data.

The program shows use of logit, probit, cloglog, loglog and cauchyit links.


\vspace*{0.2in}

\twof{beetles01.eps}{beetles02.eps}{3.1in}{3.7in}


\vspace*{0.2in}
\noindent  {\bf 2.  Vasoconstriction.~}
The data give the presence or absence ($y_i=1$  or $0$) of vasoconstriction
 in the skin of the fingers following
inhalation of a certain volume of air ($v_i$) at a certain average rate
($r_i$). Total number of records is 39.
The candidate models for analyzing the relationship are the
 usual { \tt  logit, probit, cloglog, loglog,} and {\tt cauchyit}  models.

Data are given as follows (a startup file {\tt vasoconstriction0.odc} is
also applied to save you some time):

\begin{verbatim}
  y:  1,1,1,1,1,1,0,0,0,0,0,0,0,1,1,1,1,1,
  0,1,0,0,0,0,1,0,1,0,1,0,1,0,0,1,1,1,0,0,1

  v: 3.7, 3.5, 1.25, 0.75, 0.8, 0.7, 0.6, 1.1, 0.9, 0.9,
  0.8, 0.55, 0.6, 1.4, 0.75, 2.3, 3.2, 0.85, 1.7, 1.8,
  0.4, 0.95, 1.35, 1.5, 1.6, 0.6, 1.8, 0.95, 1.9, 1.6,
  2.7, 2.35, 1.1, 1.1, 1.2, 0.8, 0.95, 0.75, 1.3

  r: 0.825, 1.09, 2.5, 1.5, 3.2, 3.5, 0.75, 1.7, 0.75,
  0.45, 0.57, 2.75, 3, 2.33, 3.75, 1.64, 1.6, 1.415,
  1.06, 1.8, 2, 1.36, 1.35, 1.36, 1.78, 1.5, 1.5, 1.9,
  0.95, 0.4, 0.75, 0.3, 1.83, 2.2, 2, 3.33, 1.9, 1.9, 1.625
\end{verbatim}

(a) Transform covariates $v$ and $r$ as
\ba
x_1 = \log(10 \times v),~~~~x_2 = \log(10 \times r).
\ea

(b) Using WinBUGS estimate posterior
 means for coefficients in the logit model.
 Use noninformative priors on all coefficients.


(c) For a subject with $v=r=1.5$, find the probability of
  vasoconstriction.


(d) Which of the five links: logit, probit, cloglog, loglog, and cauchyit,
has smallest deviance. An example for use of the five links is
provided: {\tt beetles1.odc}. Uncomment and run one link at the time.





\vspace{0.3in}

 \noindent {\bf 3. Caesarean Delivery: Categorical Response.~}
Table \ref{tab:cs} contains grouped data on infection of mothers after a C-section, collected at the
Clinical Center of the University of Munich.
\footnote{Tutz, G. (2000). Die Analyse kategorialer Daten - eine anwendungsorientierte Einführung in Logit-Modellierung und kategoriale Regression. Oldenbourg-Verlag.}
The response variable has three categories: Infection of type I, Infection of type II, and No infection.
For each mother three covariates are collected:

$$ \mbox{NOPLAN }= \left\{ \begin{array}{ll} 1 & \mbox{{~~~}C-section was not planned,}\\
   0 & \mbox{{~~~}Planned.} \end{array}\right. $$


$$ \mbox{RISK }= \left\{ \begin{array}{ll} 1 & \mbox{{~~~}Risk factors present,}\\
   0 & \mbox{{~~~}No risk factors.} \end{array}\right. $$


$$ \mbox{ANTIB }= \left\{ \begin{array}{ll} 1 & \mbox{{~~~}Antibiotics given as prophylaxis,}\\
   0 & \mbox{{~~~}No antibiotics given.} \end{array}\right. $$


\begin{table}[h]
\centering
\caption{\small Data on infections for 251 C-sections.\label{tab:cs} }
\vspace*{0.1in}
\begin{tabular}{llllllll}
\hline
  & \multicolumn{3}{l}{C-section} &  & & & \\
  \cline{2-8}
    & \multicolumn{3}{l}{Planned} & & \multicolumn{3}{l}{Unplanned}  \\
   \cline{2-4}
   \cline{6-8}
   & \multicolumn{3}{l}{Infection} & & \multicolumn{3}{l}{Infection}  \\
   \cline{2-4}
   \cline{6-8}
   & I & II & No &   & I & II &  No \\
\hline
Antibiotics  & & &  & ~~~~~ & & &  \\
{~~~~~}Risk factor & ~~0 & ~~1 & 17 &  & ~~4 & ~~7 & 87 \\
{~~~~~}No risk factor & ~~0 & ~~0 & ~~2 &  & ~~0 & ~~0 & ~~0 \\
\hline
No Antibiotics  & & &  & ~~~~~ & & &  \\
{~~~~~}Risk factor & 11 & 17 & 30 &  & 10 & 13 & ~~3 \\
{~~~~~}No risk factor & ~~4 & ~~4 & 32 &  & ~~0 & ~~0 & ~~9 \\
\hline
\end{tabular}
\end{table}

(a) Given the covariates, establish a multinomial model, where the outcome 'No infection' serves as a baseline.


(b) A new C-section delivery for a mother
 with covariates (NOPLAN, RISK, ANTIBIO)=(1,0,0) is to be evaluated for risks of infection.
 What are estimated probabilities of no infection, and type I and II infections.

\vspace*{0.1in}
\noindent {\sl Hint:} ~Consult {\tt NHANESmulti.odc} discussed in UNIT7

\vspace*{0.3in}

\noindent {\bf 4. Magnesium Ammonium Phosphate and Chrysanthemums.~}
%
Walpole et al.\ (2007)  provide data from a study
on the effect of magnesium ammonium phosphate on the height of chrysanthemums,
which was conducted at George Mason  University in order to determine a possible
optimum level of fertilization, based on the enhanced vertical growth
response of the chrysanthemums. Forty chrysanthemum seedlings were assigned
to 4 groups, each containing 10 plants. Each was planted in a similar pot
containing a uniform growth medium. An increasing
concentration of MgNH$_4$PO$_4$, measured in grams per  bushel, was added to each plant. The 4
groups of plants were grown under uniform conditions in a greenhouse for a
period of 4~weeks. The treatments and the respective changes in heights,
measured in centimeters, are given in the following table:

\begin{center}
\begin{tabular}{|r|r|r|r|}
\hline
\multicolumn{4}{|c|}{Treatment} \\
\hline
~50 g/bu & 100 g/bu & 200 g/bu & 400 g/bu\\
\hline
13.2~ & 16.0~ &  7.8~ & 21.0~ \\
12.4~ & 12.6~ & 14.4~ & 14.8~ \\
12.8~ & 14.8~ & 20.0~ & 19.1~ \\
17.2~ & 13.0~ & 15.8~ & 15.8~ \\
13.0~ & 14.0~ & 17.0~ & 18.0~ \\
14.0~ & 23.6~ & 27.0~ & 26.0~ \\
14.2~ & 14.0~ & 19.6~ & 21.1~ \\
21.6~ & 17.0~ & 18.0~ & 22.0~ \\
15.0~ & 22.2~ & 20.2~ & 25.0~ \\
20.0~ & 24.4~ & 23.2~ & 18.2~ \\
\hline
\end{tabular}
\end{center}

Solve the problem as a Bayesian one-way ANOVA. Use STZ constraints
on treatment effects.

(a) Do different concentrations
of MgNH$_4$PO$_4$ affect the average attained height of chrysanthemums? Look at the 95\% credible sets for the differences between treatment effects.

(b) Find the 95\% credible set for the contrast $\mu_1 - \mu_2 - \mu_3 + \mu_4.$




  \vspace*{0.3in}
\noindent {\bf 5. Third-degree Burns.}~ %===============================================
%
The data for this exercise, discussed in Fan et al. (1995), refer to $n=435$ adults who were treated for third-degree burns by the University of Southern California General Hospital Burn Center. The patients were grouped according to the area of third-degree burns on the body. For each midpoint of the groupings ``log(area +1),'' the number of patients in the corresponding group who survived and the number who died from the burns was recorded:


\begin{center}
\begin{tabular}{lcc}
\hline
Log(area+1) & Survived & Died \\
\hline
1.35 & 13 & ~~0\\
1.60 & 19 & ~~0\\
1.75 & 67 & ~~2 \\
1.85 & 45 & ~~5 \\
1.95 & 71 & ~~8 \\
2.05 & 50 & 20\\
2.15 & 35 & 31 \\
2.25 & ~~7 & 49 \\
2.35 & ~~1 & 12\\
\hline
\end{tabular}
\end{center}

(a) Fit the logistic
regression on the probability of death due to third-degree burns with
the  covariate {\tt x = log(area+1)}. What is the deviance?


(b) Using your model, estimate find the posterior probability of survival for a person for which {\tt log(area + 1)} equals 2.


(c) Repeat (a) with probit and complemenry log-log links.
In terms of deviance, which morel provides the best fit.

  \vspace*{0.3in}
\noindent {\bf 6.  Shocks!~}
An experiment was conducted to assess the effect of
small electrical currents on farm animals, with the eventual goal
of understanding the effects of high-voltage powerlines on
livestock.  The experiment was carried out with seven cows, and
six shock intensities, 0, 1, 2, 3, 4, and 5 milliamps (shocks on
the order of 15 milliamps are painful for many humans.\footnote{ C. F.
Dalziel, J.B. Lagen and J. L. Thurston, {\it Electric shocks},
\textit{Trans IEEE} \textbf{60} (1941), 1073-1079. }  Each cow was
given 30 shocks, five at each intensity, in random order.  The
entire experiment was then repeated, so each cow received a total
of 60 shocks.  For each shock the response, mouth movement, was
either present or absent.  The data as quoted give the total
number of responses, out of 70 trials, at each shock level.  We
ignore cow differences and differences between blocks
(experiments).

\begin{center}
  \begin{tabular}{cccc}                                             \hline
    Current         & Number of     & Number of  & Proportion of \\
    (milliamps) $x$ & Responses $y$ & Trials $n$ & Responses $p$ \\ \hline
          0         &       0       &     70     &     0.000     \\
          1         &       9       &     70     &     0.129     \\
          2         &      21       &     70     &     0.300     \\
          3         &      47       &     70     &     0.671     \\
          4         &      60       &     70     &     0.857     \\
          5         &      63       &     70     &     0.900     \\ \hline
  \end{tabular}
\end{center}
As in Exercise Beetles (Bliss Data) model $y$ as a function of $x$ via binary regression with 5 different links and propose the link that minimizes the deviance.

 \vspace*{0.3in}
\noindent {\bf 7.  Three WinBUGS Programs and GLM Practice.~}
Three WinBUGS programs are supplied: iop2.odc, NBreg.odc, and 
terrapins.odc.

Read intro to the programs (preables in ODC files) and run them on your computer.
Understand the output.


\end{document}
